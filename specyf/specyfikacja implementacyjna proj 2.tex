\documentclass[12pt,a4paper]{article}

\usepackage{polski}
\usepackage{graphicx} 
\usepackage[utf8]{inputenc}

\title{Specyfikacja implementacyjna generatora tekstu}
\author{Konrad Olszewski}

\begin{document}
\maketitle
\tableofcontents


\section{Język}
\\
Program będzie napisany w języku c.
\section {Opis}
\subsection{Użytkownik docelowy.}
\\
Pogram dedykowany jest studentom oraz pracownikom Politechniki Warszawskiej.

\section{Podział na klasy - opis}

\subsection{Opis klas i metod}
\\
Program zostanie podzielony na następujące klasy:
\begin{enumerate}
\item Ngram.class - klasa będąca pojedynczym ngramem danego rzedu prefix (String ngram) + sufixy(String []nastepcy) + ilość wystąpień danego sufixu(ileRazy[])

\begin{itemize}
	\item rozszerzNgram(String slowo) - rozszerza Ngram o dodatkowy sufix
	\item NgramRandom() - zwraca losowy sufix Ngramu
	\item NgramSuma() - zwraca ilość wysąpień Ngramu
\end{itemize}
\\
\item
Read.class - klasa wczytujaca plik bazowy, generuje ona listę ngramów z pliku z tekstem, bądź ze Stringa. Zawiera listęNgramów danego stopnia, listęngramów jednoskowych, poziom ngramu, oraz ścieżkę do bazy

\begin{itemize}
	\item String czytaj(String path) - klasa czyta plik który dostaje i zwraca go w formie Stringa
	\item przetwórz(ArrayList $\langle Ngram \rangle$ ngramy,String in, Integer ngramLvl) - Pełną listę słow (in) dzieli na ngramy, następnie sprawdza zawieranie kolejnych ngramów w już istniejącej ich liście. Po sprawdzeniu czy istnieje dodaje nowy Ngram do listy lub rozszerza istniejacy ngram.
	\item Sortuj() - sortuje listę nramów po ilości wystąpień
\end{itemize}
\\ 
\item
Gui.class - klasa generujaca interfejs graficzny 
\begin{itemize}
	\item public void actionPerformed( ActionEvent e) - klasa odpowiadajaca za akcje
\end{itemize}
\\

\item
Txtgenerator.class - klasa generujaca tekst zadanej dlugosci na podstawie listy ngramow oraz odpowiedzi użytkownika.
\item
Statistics.class - klasa generujaca statystyki tekstow. Zawiera 
\begin{itemize}
	\item 
String najczestszeWyrazy()
\item
String najczestszeSlowa()
\item
String najczestsze(String in, List $\langle Ngram \rangle$ ngramy) - wyznacz listę n najpopularniejszych, w zalezności od listy, ngramów/ słów
\end{itemize}
\end{enumerate}
\subsection {Opis GUI}
Gui będzie podzielone na dwie części. Główne okno czatu które wygląda jak na Załączonym obrazku g1.png.
Po prawej stronie będzie widoczna część statystyki. Tutaj wyświetlana będzie bieżąca statystyka tekstu.
\section{OPIS FUNKCJONALNOŚCI}
\subsection{Jak korzystać z programu?}
\\
Aby korzystac z programu nalezy wczytac plik bazowy.
Nastepnie pojawia sie mozliwosc wpisywania wiadomosci na ktore uzyskiwane beda odpowiedzi w oknie chatu.\\
Po prawej stronie okna wyświetlana będzie aktualna statystyka tekstu.
\\
Ponadto program będzie posiadał nastepujace funkcje:
\\
\begin{enumerate}
\item
wybrania nicku
\item
wychwycenie błędów we wpisywanym tekście 
\end{enumerate}
\section{SCENARIUSZ DZIAŁANIA PROGRAMU.}
\subsection{Scenariusz ogólny:}
\begin{enumerate}
\item
Uruchomienie programu
\item
Wprowadzenie nicku
\item
Wprowadzenie wiadomości
\item 
Uzyskanie wygenerowanej odpowiedzi
\item
Uzyskanie wygenerowanych statystyk
\end{enumerate}
\\

\section{TESTOWANIE.}

\subsection{Ogólny przebieg testowania:}
\\
Program będzie testowany za pomoca programu eclipse, który będzie znajdował poszczególne błędy. Będzie również testowany na plikach różnej wielkości.
Program zostanie również przetestowany za pomocą testów jednostkowych JUnit.
Przykład testu dla klasy Read.
\\
public class ReadTest {
\\
	\\@Test
	\\public void testRead() {
		\\String tekst = "Alibaba i czterdziestu rozbójników buszuj. Alibaba i 12";
		\\Read r = new Read( tekst, true);
		\\assertEquals(r.listaNgram.get(0).slowo, "Alibaba i");
\\		assertEquals(r.listaNgram.get(1).slowo, "i czterdziestu");
	\\	assertEquals(r.listaNgram.get(2).slowo, "czterdziestu rozbójników");
		\\assertEquals(r.listaNgram.get(3).slowo, "rozbójników buszuj.");
\\		assertEquals(r.listaNgram.get(4).slowo, "buszuj. Alibaba");
\\
	\\	assertEquals(r.listaNgram.get(0).nastepcy[0], "czterdziestu");
		\\assertEquals(r.listaNgram.get(0).nastepcy[1], "12");
		\\
	\\	assertEquals(r.listaNgram.get(1).nastepcy[0], "rozbójników");
	\\	assertEquals(r.listaNgram.get(2).nastepcy[0], "buszuj.");
	\\	assertEquals(r.listaNgram.get(3).nastepcy[0], "Alibaba");
	\\	assertEquals(r.listaNgram.get(4).nastepcy[0], "i");
		
\\	}
\section{DIAGRAM UML}
\begin{figure}[h]
	\centering
		\includegraphics[width=1\textwidth]{C:/Users/ja/Desktop/diagramUML.png}
	\label{fig:diagramUML}
\end{figure}



\end{document}